\begin{abstractpage}
Intraplate faulting in central Asia is a major component of the Indo-Eurasian collision. The kinematics and mechanisms of intraplate deformation are important in understanding broad active tectonic patterns, reconstructing past tectonics, analyzing seismic hazard and identifying potential resources. We examine the fault kinematics surrounding the 150km wide Shargyn Basin at the intersection of the left-lateral transpressional Gobi-Altai fault system and the right-lateral transpressional Altai fault system. The studies were performed using satellite data and targeted field transects. The results suggest the Shargyn basin is formed by a compressional stepover, an uplifted wedge from the intersecting strike-slip systems and many terminal thrust splays. The fault-kinematics, pre-existing structural weaknesses and a significant gravity anomaly all suggest that the basin basement is mechanically strong. This research demonstrates some of the potential kinematics for intraplate transpressional orogenies and emphasizes the importance of pre-existing crustal structure in the development of active faults.
\end{abstractpage}
