\section{Geophysical Data}
	The lack of faulting in the Shargyn Basin is anomalous and could be due to unique basement geology. I examined the available gravity, magnetic and seismic data for the region to derive constraints on the subsurface geology in the Shargyn Basin.

	Bouguer gravity anomaly measurements (shown in Figure \ref{gravity}) indicate that the crust beneath the Shargyn Basin is denser than the surrounding area. Bouguer gravity anomalies are deviations from the geoid gravity field after correcting for the elevation of the measurement and the excess mass above the geoid. A deep crustal root implies lower density and thus, isostasy should cause large elevated regions to show a large negative bouguer anomaly, whereas topographically low regions should show a more positive bouguer anomaly. This can be seen in Figure \ref{gravity}, where the Altai mountain range and the Hangai mountain ranges show very strong negative Bouguer anomalies. The topographically low region south of the Altai and Gobi-Altai and the "Lakes" region, between the Altai and the Hangai both show significantly less negative anomalies, as expected for topographic lows.
	
	Bouguer gravity anomaly measurements were calculated from the EGM2008 global gravity model \citep{EGM2008}, which uses a spherical harmonic expansion through order 2160. Compared to the surrounding mountain ranges, the Shargyn Basin shows a more negative bouguer anomaly. Furthermore, compared to the rest of the region, many of the mountain ranges immediately surrounding the Shargyn Basin show a surprisingly small negative bouguer anomaly. The negative bouguer anomaly beneath the Shargyn Basin can be explained by a denser crustal basement. The small negative bouguer anomaly in the surrounding mountain ranges may indicate lighter than average crustal densities in these regions. \citet{Bayasgalan2005a} calculates an average elastic thickness for Mongolia of ~6km. This elastic thickness would lead to almost complete isostatic adjustment for an area the size of the Shargyn Basin. Hence, the anomaly probably cannot be attributed to isostatic adjustment from other nearby anomalies, like the Sutai Range. The density anomalies could be linked to the location of past and present faulting. Perhaps, the denser basement under the Shargyn Basin is more difficult to fracture and deform.

\begin{figure}[h!]
  \centering
  \includegraphics[width = 5in, height = 5.0in]{gravity.png}
  \caption{Bouguer gravitational anomalies in western Mongolia. The Shargyn Basin is outlined in black. Compare the more negative anomalies under the Shargyn Basin to the more positive anomalies underneath the surrounding mountain ranges. This is the opposite of what we would expect from isostatic considerations.}
  \label{gravity}
\end{figure}

	Magnetic anomalies could also indicate a different composition of crust beneath the Shargyn Basin, however the signal is ambiguous in the Shargyn Basin. I used the NGDC-720 Crustal Magnetic Field Model, a elliptical harmonic expansion of the gravitational field to $n = 720$\citep{Maus2010}.  The gravitational anomaly above the Shargyn Basin and surrounding areas is shown in Figure \ref{magnetic}. The Shargyn Basin is at the location of a small positive magnetic anomaly. However, the spatial extent of the anomaly does not match very well with the shape of the Shargyn Basin. This is to be expected, because the minimum resolvable wavelength for a harmonic expansion of order 720 is ~56km. It is also likely that there is significant spatial error near the minimum wavelength. The Shargyn Basin is only 140km across -- less than 3 minimal wavelengths. Therefore, we are currently unable to decide whether there is a real magnetic anomaly beneath the Shargyn Basin. 

\begin{figure}[h!]
  \centering
  \includegraphics[width = 5in, height = 5.0in]{{magnetic.ps}.png}
  \caption{Magnetic anomaly in western Mongolia. The Shargyn Basin is outlined in black. Note the much lower resolution than the gravity measurements.}
  \label{magnetic}
\end{figure}

	Earthquake focal mechanisms provide useful information about active fault slip in the region. The available calculated focal mechanisms for western Mongolia are shown in Figure \ref{broad}. The World Stress Map \citep{WorldStressMap2008} uses focal mechanisms as the exclusive constraint on the primary stress axis in western Mongolia due to the lack of other data. \citet{Bayasgalan2005a} provides a good review of the earthquake focal mechanisms in western Mongolia. On the Shargyn Fault, near cross-section C-C', a focal mechanism was calculated showing exclusively left-lateral slip. In the northern corner of the intersection wedge, a focal mechanism was calculated either showing low-angle south directed thrusting or high-angle north directed thrusting. Both possibilities are reasonable given the location of the earthquake. 

\begin{figure}[h!]
  \centering
  \includegraphics[width = 5in, height = 2.88in]{eq_locs.png}
  \caption{Shaded relief map of the Shargyn Basin showing blue plus signs where earthquakes have occured between 1960 and 2013. Black lines are fault locations, like in Figure \ref{faultmap}.}
  \label{eqlocs}
\end{figure}	
	Earthquake locations can also constrain the location of active deformation. For smaller earthquakes, focal mechanisms cannot be calculated because there is a lack of data. However, locations can still be determined, albeit with significantly decreased accuracy. The locations for all the earthquakes available from the IRIS Data Management System in the Shargyn Basin region are shown in Figure \ref{eqlocs}. Most of these earthquakes occur around the margins of the basin at locations near where we have mapped faults. Notably, there are three earthquakes shown in the eastern part of the Shargyn Basin. These earthquakes might indicate incipient faulting in the basin.

	Seismic wave speeds underneath the basin could indicate different crustal properties. I investigated available P-wave and S-wave global teleseismic models, but no existing models have sufficient resolution to determine the wave speed anomaly beneath the Shargyn Basin. If there have been local seismological surveys performed, I am unaware of them. 


