\section{Introduction}
	Continental deformation can occur far from plate boundaries. The behavior of intraplate fault systems is poorly understood. Furthermore, where multiple fault systems intersect, complex kinematics can occur. Understanding how and why these kinematic patterns develop is important for understanding broad active tectonic patterns, reconstructing past tectonics, analyzing seismic hazard and identifying potential resources. 

	In this paper, we examine the intersection of the Altai Range and the Gobi-Altai Range. The Altai Range and the Gobi-Altai Range are two large, actively forming mountain ranges in central Asia. The Altai Range is characterized by right-lateral transpressional slip, while the Gobi-Altai Range is characterized by left-lateral transpressional slip \citep{Cunningham2005a}\citep{Cunningham2010}. The intersection of these two fault systems provides a natural laboratory for studying strike-slip kinematics in an area with a long, varied tectonic history. Furthermore, an understanding of this critical zone will help elucidate the evolution of regional tectonic patterns.

	The 150 km wide Shargyn Basin in western Mongolia lies at the intersection of the Altai fault system and the Gobi-Altai fault system. Here, we describe the Shargyn Basin as the result of transpressional strike-slip kinematics. The kinematics consist of a compressional strike-slip stepover, a conjugate strike-slip intersection, and termination thrust splays. We also discuss the active faulting in the context of pre-existing foliation and structural weaknesses.

