\section{Motivation}
	Continental deformation can occur far from plate boundaries. The behavior of intraplate fault systems is poorly understood. Furthermore, where multiple fault systems intersect, complex kinematics can occur. Understanding how and why these deformation patterns develop is important for understanding broad active tectonic patterns, reconstructing past tectonics, analyzing seismic hazard and identifying potential resources. 

	In this paper, I examine the intersection of the Altai Range and the Gobi-Altai Range. The Altai Range and the Gobi-Altai Range are two large, actively forming, mountain ranges in central Asia. The Altai Range is characterized by right-lateral transpressional slip, while the Gobi-Altai Range is characterized by left-lateral transpressional slip \citep{Cunningham2005a}\citep{Cunningham2010}. The intersection of these two fault systems provides a natural laboratory for studying strike-slip kinematics in an area with a long, varied tectonic history. Furthermore, an understanding of this critical zone will help elucidate the evolution of regional tectonic patterns.
	
\begin{figure}[h!]
	\centering
	\includegraphics[width = 6.5in, height = 4in]{broad.png}
	\caption{Shaded relief map of Mongolia with fault overlays and seismic moment tensor solutions for recorded earthquakes. Important locations or faults are labeled. The study area and location of Figure \ref{regional} is outlined in black. The difference between right-lateral transpressional strike-slip in the Altai Range and left-lateral transpressional strike-slip in the Gobi-Altai is especially important. Modified after \citet{Calais2003}}
	\label{broad}
\end{figure}

\begin{figure}[h!]
	\centering
	\includegraphics[width = 6in, height = 4in]{regional.png}
	\caption{Shaded relief map showing notable features in the Shargyn Basin region. Many of these faults are studied in this paper. GPS velocity magnitude for the Altai site is 4.4mm/yr. Faults are shown. See Figure \ref{faultmap} for the direction of slip.}
	\label{regional}
\end{figure}
	
	The 150km wide Shargyn Basin in western Mongolia lies at the intersection of the Altai fault system and the Gobi-Altai fault system. Here, I describe the Shargyn Basin as the result of transpressional strike-slip kinematics. The kinematics consist of a compressional strike-slip stepover, a conjugate strike-slip intersection and termination thrust splays. I also use gravity data to argue that a difference in basement rock has been a primary control on the neotectonic development of the Shargyn Basin region.


