\section{Structural Evidence}
The Shargyn Basin is a combination of a conjugate strike slip fault intersection and a left-lateral compressional strike-slip stepover. The Bogd Fault to the south, the Tonhil Fault to the west, the Shargyn Fault to the north and the Khantaishir Thrusts to the east are the primary faults governing the structure of the Shargyn Basin. The Bogd Fault and the Shargyn Fault are linked in a compressional stepover by the Khantaishir Thrusts. However, some motion remains on the continuation of the Bogd Fault. The remainder of this motion is transferred into multiple thrust splays, the last of which combines with terminal thrust splays from the Tonhil Fault to form an uplifted region that I will call the intersection wedge. Similar thrust splays on the Shargyn Fault uplift the Dariv Range and link with thrust faults in the Dariv Basin and the Sutai Range. Figure \ref{schematic} shows a schematic diagram of this structure.

\begin{figure}[h!]
  \centering
  \includegraphics[width = 6in, height = 3.5in]{schematic.png}
  \caption{Schematic structural fault map of the Shargyn Basin region. The extent is approximately the same as the extent in Figure \ref{regional}. Description of the structure is in the text. The shaded areas are basins. Unshaded areas are mountainous. The dotted lines represent the dominant foliation strike.}
  \label{schematic}
\end{figure}

	Thus, the eastern part of the Shargyn Basin is a stepover basin forming from the Khantaishir Thrusts. The western and southern part of the basin is a foreland basin from the uplifted intersection wedge and the many termination thrust splays of the Bogd Fault. On the northern boundary, the uplift of the Dariv Range is concentrated on the Dariv Thrust and the Shargyn Fault separates the low-lying basin from the adjacent mountain range while not accomodating the uplift.
\subsection{Remote Sensing}
Previous studies have examined Mongolian active tectonics from a remote sensing perspective (cite some of the tapponnier papers and the cunningham papers and walker papers). I performed a more detailed remote sensing survey of the Shargyn Basin region using Landsat 7 TM+ satellite imagery. Surface features indicating active motion were identified. Because the region is arid, many important features are clearly visible. For example, the trace of Tonhil Fault on the western margin of the Shargyn Basin is visible as a very sharp linear topographic feature. Using such features from Landsat imagery was useful in identifying important features for further investigation in the field. Also, using contiguous features on the satellite imagery, field mapping results can be extended to nearby, unobserved areas.

In the remote sensing studies, I focused on identifying prominent faults using steep topography, sharp color changes (lithology changes), fault scarps, and offset features. I also mapped clearly distinguishable lithologic units using color change in the Landsat imagery. Where possible, I noted the strike of the foliation. Foliation is useful to map out the Paleozoic structural grain in the region. I also noted the location and distribution of alluvial fans. Locations that have alluvial fans forming on top of mountain ranges are important because that might indicate a time difference in the formation or slip on multiple faults.

\subsection{Field Mapping}
During the summer of 2012, Claire Bucholz, myself, and Esoo Erdene, a student at Mongolian University of Science and Technology performed two months of fieldwork in the Shargyn Basin region. We spent part of the time focusing on the active faults. We performed 10-20km transects through areas with important faults, as previously identified by satellite image (mostly similar to the faults identified by \citep{Walker2007}). We mapped at a broad scale, but made careful note of features indicating recent brittle deformation. The level of fracturing and fault exposures were important in locating faults. Where no other evidence is available, a very steep topographic step is associated with fault. Mapping was performed on paper with the assistance of satellite imagery and a GPS device. 

The results of this mapping are summarized in a fault map in Figure \ref{faultmap}. Published mapping in the Sutai Range by \citet{Cunningham2003} and in the Dariv Range by \citet{Dijkstra2006} is incorporated. Further unpublished mapping of the Dariv Range by Claire Bucholz is incorporated. We performed new mapping surrounding the Shargyn Basin Eleven cross-sections were created. Six of these are shown in the text. The remaining cross-sections are in Appendix A. The following sections describe in detail the important structural results, while referencing these cross-sections.
 
\begin{figure}[h!]
  \centering
  \includegraphics[width = 7in, height = 5.0in]{faultmap.png}
  \caption{The black lines are faults with the indicated slip direction and dip direction (for thrusts). }
  \label{faultmap}
\end{figure}

\subsection{Khantaishir Thrusts -- Northwards Slip Transfer}
\begin{figure}[h!]
  \centering
  \includegraphics[width = 6in, height = 1.8in]{khantaishir.png}
  \caption{Northeast-Southwest cross-section through the Khantaishir Range and the Khantaishir Thrusts.}
\end{figure}


\begin{figure}[h!]
  \centering
  \includegraphics[width = 6in, height = 1.8in]{khantaishir_sat.png}
  \caption{Landsat imagery showing the Khantaishir Range and the mapped active faults.}
\end{figure}

The eastern edge of the basin is uplifted with motion concentrated on two fault which I call the Upper Khantaishir Thrust (UKT) and the Lower Khantaishir Thrust (LKT). The cross-section across these two faults is marked on the map as E-E'. Other minor faults are identifiable on the satellite imagery. We did not investigate the minor faults in the field. Both the UKT and LKT strike approximately NNW-SSE. The UKT is the northern fault. On the northeast side of the UKT, we observed a thick fractured layer of volcaniclastics underneath a sedimentary sequence with a thick conglomerate layer followed by shales, sandstones and carbonate reef deposits. At higher elevations in the mountain range, mafic and ultramafic ophiolite-related rocks are present. No outcrop is found on the southwest side of the UKT. At the alluvium-outcrop transition, we observed a 3-5m wide deformation zone that could be interpreted as the UKT itself. This zone is shown in Figure \ref{faultzones}A. Nearing this thrust trace, there is a significant increase in fracturing with strikes between 280$\circ$ and 320$\circ$. Fracture dip was similar to the dip of the UKT itself. Rotated sigmoidal tension gashes were observed immediately above the deformation zone indicating a thrust sense of motion. S-C fabric within the deformation zone also indicated thrust motion.

\begin{figure}[h!]
  \centering
  \includegraphics[width = 4in, height = 6in]{faultzones.png}
  \caption{Photographs of exposed faults. Field workers are used as scale bars. A) An outcrop of the exposed Upper Khantaishir Thrust from cross-section E-E'. B) An outcrop of the exposed thrust in cross-section G-G'}
  \ref{faultzones}
\end{figure}


The trace of the Lower Khantaishir Thrust (LKT) is located 10km to the southwest of the UKT. South from the UKT, there is a transition from recent alluvium to more lithified, potentially older alluvium. Then, for 2 kilometers before reaching the LKT, there is a thick stack of sandstone and basalt flows, with interlayered sandy carbonate beds. Finally, there is a very sharp topographic transition and lithologic shift to recent alluvium. This marks the fault trace for the LKT. The sedimentary sequence norteast of the LKT indicates the presence of a minor basin in the area at a point after the widespread metamorphism affecting the area (CITE the russian map which called it devonian?). The older alluvium further northwest also support a complex faulting history.

There is evidence that the motion on the LKT began after the motion on the UKT. In the small uplifted region near the trace of the LKT, there are deeply incised, but highly sinuous incised drainage canyons. Such incised drainages indicate that, before uplift began, a pre-existing sinuous drainage pattern existed. The uplift was rapid enough that the drainage pattern was "locked in". Because these drainages are aligned with the current southwestwards drainage patterns from the Khantaishir Range and the UKT, it fits with a southwestwards drainage pattern for the pre-uplift phase. Southwestwards drainage in the pre-uplift phase is compatible with already ongoing uplift on the UKT.

\subsection{Shargyn Fault and Termination Thrust Splays}

The northern margin of the basin is separated from the Dariv Range by the Shargyn Fault. Four of the transects focus on this fault. The fault strikes WSW-ENE. I have assumed that the fault is approximately vertical because the motion is strike-slip. 

\begin{figure}[h!]
  \centering
  \includegraphics[width = 6in, height = 9.0in]{shargynfault.png}
  \caption{Cross sections through the Shargyn Fault and its termination splays.}
  \label{shargynfaultxsecs}
\end{figure}
\clearpage
The western end of the Shargyn Fault, nicknamed Little Dariv Mountain, exhibits thrusts in a small positive flower structure, seen in cross-section A-A' in Figure \ref{shargynfaultxsecs}. The positive flower structure has been identified by previous researchers \citep{Howard2006}. Their interpretation is similar to the one discussed here. However, there are differences in the number and location of thrusts. To the north of the Shargyn Basin is the Dariv Basin, a much smaller, higher elevation basin. Little Dariv Mountain is located on the boundary of these two basins. The majority of the range is composed of greenschist-facies sedimentary and volcanic rocks. On the northern side of the range, there is a 100 meter thick band of folded carbonates and marls. Separating the range is a thrust fault. The fault follows a valley that is clearly visible on satellite imagery. To the south of the fault is a steep 200 meter topographic rise, indicating that the fault dips southwards. No other faults were identifiable in the core of Little Dariv Mountain. 

On the cross-section, two other faults are indicated on the periphery of the mountain. The southern fault can be seen in the satellite imagery as an uplifted trace of bedrock amongst a sea of basin sediments. It may be related to a large anticline visible on the satellite imagery west of Little Dariv Mountain. The northern boundary fault is the inferred continuation of the main Shargyn Fault trace.  This continuation of the Shargyn Fault terminates as thrust motion on thrusts in the Dariv Basin. Because the Dariv Basin thrusts are the easternmost extent of the Sutai Range restraining bend \citep{Cunningham2003}, the thrusting involved in the Sutai Range restraining bend can be linked with the strike-slip motion on the Shargyn Fault. 

In the central south of the Dariv Range, we examined a simple section of the Shargyn Fault, marked B-B' in Figure \ref{shargynfaultxsecs}. To the north, the topography rises steeply towards a relatively flat plateau. The bedrock is composed of serpentinite, biotite-gabbro and gabbro-diorite, separated by intrusive contacts. Despite the steep topographic rise at the fault, it is not likely there is much thrusting on the Shargyn Fault here. Thrusting on the Shargyn Fault at this location would conflict with the lack of topographic expression along the fault east of the Dariv Range. It is more likely that the Dariv Thrust to the northeast is uplifting the range and the Shargyn Fault is simply accomodating the juxtaposition of the uplifted Dariv Range and the Shargyn Basin. 

The simplest cross-section of the Shargyn Fault is in the northeastern part of the basin, marked on the map as C-C' and shown in Figure \ref{shargynfaultxsecs}. The Shargyn fault is visible in satellite imagery of the cross-section area as a shift in topographic expression (REFERENCE IMAGE). It also marks a shift from a black unit on the south side to a dark and light brown unit on the north side. Slightly to the west of the cross-section, in the basin, there are clearly visible sinistral stream offsets (Figure \ref{shargynoffsetstreams}). Field observations indicate that the southern unit is primarily a quart-rich greenschist with a northwards dipping foliation. There are also minor components of metamorphosed conglomerate with preserved clasts and carbonate-rich rocks, especially closer to the fault.  The fault itself passes through a topographic low and does not outcrop. On the north side of the fault, we see a higher-grade gneissic unit with steeper, south dipping foliation. Further north, we found an outcrop of garnet schist, confirming the shift in metamorphic grade. The topography and lithology shifts both provide evidence for the presence of a fault at the marked location. The topography (maybe show a picture?) and the offset stream beds (definitely show an image of the offset stream beds) located to the west imply that the Shargyn Fault is actively deforming. The slight southward dip of the fault was not observed but is inferred based on the elevated topography on the southern side of the fault. 

\begin{figure}[h!]
  \centering
  \includegraphics[width = 4in, height = 2.0in]{shargyn_fault_streams.png}
  \caption{Offset stream beds along the ENE-WSW striking Shargyn Fault observed in Landsat imagery west of cross-section C-C'.}
  \label{shargynoffsetstreams}
\end{figure}

To the east, the strike of the Shargyn fault bends slightly southwards, at the cross-section marked D-D' in Figure \ref{shargynfaultxsecs}. Here, the fault is still noticeable in satellite imagery (REFERENCE IMAGE). To the south of the fault, we see steep rolling hills with highly deformed and metamorphosed amphibolite and gneiss. To the north of the fault, the topography is lower and transitions into alluvium. The northern unit is an slightly deformed alkali granite (show picture that I took?). The difference in metamorphic grade between the granite and the metamorphic unit indicates the granite was intruded after metamorphism. However, the contact between the granite and the metamorphic is perfectly straight, supporting the contact's interpretation as a fault. The topographic depression along the fault provides further evidence. To the east, the fault terminates somewhere near the city of Altai. 

West of the transect, a thrust splays off the Shargyn Fault striking southwards. The thrust intersects the transect in the south between a red very poorly consolidated sandy unit and a steeply dipping shale. The northern side of the thrust is uplifted, while the basin is on the south side. Further south, elevation profiles suggest there may be much more thrust motion. Further south, this thrust connects with the Khantaishir Thrust system. 

\subsection{Tonhil Fault and Intersection Wedge Thrusts}
Further west, the wedge between the Tonhil and Gobi-Altai Faults exhibits some of the most extensive thrusting. 

\begin{figure}[h!]
  \centering
  \includegraphics[width = 6in, height = 6.0in]{intersectionwedgexsecs.png}
  \caption{Cross sections through the intersection wedge.}
  \label{intersectionwedgexsecs}
\end{figure}

\begin{figure}[h!]
  \centering
  \includegraphics[width = 6in, height = 1.8in]{intersection_wedge_sat.png}
  \caption{Landsat imagery showing the intersection wedge area at the southwestern corner of the Shargyn Basin. Mapped active faults are shown by dashed lines Note the two conjugate strike-slip faults, the stacked thrusts at the basin margin and the two small transtensional basins located at the eastern and northern corners of the wedge.}
\end{figure}


At the far southern corner of the thrust wedge, we examined the terminus of the Bogd Fault, shown in cross-section J-J' in Figure \ref{intersectionwedgexsecs}. After the various thrust splays further east and north, it is unlikely that there is significant relative motion so far south. However, what remains is directed into a thrust fault that strikes approximately 45 degrees off the main Bogd Fault. There is a noticeable topographic rise bending away form the main Bogd Fault. The small depositional region and the topographic rise would be unlikely without recent motion. In addition, there was a higher local degree of damage to the bedrock. We observed a fracture set at (FILL IN S/D). 

At the eastern corner of the wedge, there is a bend in the Gobi-Altai Fault from E-W striking to ENE-WSW striking. At the location of this bend, there is a small (2km wide) transtensional basin, shown in cross-section H-H'. This has been previously identified on satellite imagery by \citet{Cunningham2010}. We examined the area in the field. Though we were not able to directly observe any deformation structures, the southern margin of this basin had a very steep topographic rise. North of the basin, we found a large granite intrusion. South of the basin, there are south dipping high grade metamorphic rocks. Because of the topographic depression and local deposition, the small basin is probably bounded by a normal fault on one side and the left-lateral Bogd Fault on the other. 

The western boundary of the intersection wedge is controlled by the Tonhil Fault. The Tonhil Fault is a major NNW-SSE striking right-lateral strike slip fault. At the cross-section K-K', we examine the fault where it interacts with the northwestern corner of the intersection wedge. At the latitude of the cross-section, the Tonhil fault bends so that it strikes nearly N-S. This bend appears to have created another transtensional basin. We did not examine the transtensional basin in the field. However, it is interesting that it lies symmetrically across the intersection wedge from the transtensional basin on the Bogd Fault examined in cross-section H-H'. 

Further east, along cross-section K-K', we observed a boundary between greenschist grade metasedimentary rocks on the east side and a gneissic unit on the west side. This is interpreted as the furthest northern extent of the southwestern-most of the intersection wedge thrusts. For most of its length, satellite imagery of this fault shows some topographic expression and where observed in the field, it always separates the gneiss and schist from the metasedimentary rocks, indicating that it is an important structural boundary. 

ADD A SECTION ON THE MAIN TRANSECT THROUGH THE WEDGE?!

\subsection{Bogd Fault and Termination Thrust Splays}

Along the southern margin of the basin, we encounter a repeated motif where thrusts splay off the main Gobi-Altai left-lateral strike slip fault. These faults can be seen in the large scale fault map. Two transects were studied across these thrust splays. 

\begin{figure}[h!]
  \centering
  \includegraphics[width = 6in, height = 3.5in]{bogdfaultxsecs.png}
  \caption{Cross sections through the Shargyn Fault and its termination splays.}
  \label{bogdfaultxsecs}
\end{figure}

The easternmost of these thrusts can be seen in cross-section F-F' in Figure \ref{bogdfaultxsecs}. Directly above (south) the thrust, there is a complex sequence of metasedimentary and metavolcanic that is intruded by a large gabbro body. A bit to the west of the cross-section line, the topography lowers and outcrop is only visible within a short band near the fault itself. The average strike of the foliation in the fault-adjacent metasedimentary unit is visibly parallel to the fault on satellite imagery. This is confirmed in the field, though minor folding interferes locally. This suggests that the fault follows a pre-existing weakness in the meta-sedimentary unit. The elevation along the basin margin decreases to the northwest, away from the Bogd Fault. This might indicate a continuous decrease in slip. Sediment accumulating behind the thrust points to the presence of another fault. Further south along the cross-section, we encounter this fault. The lithology changes sharply from a sandstone and basalt sequence to a greenschist facies metasedimentary sequence. Furthermore, there is another jump in topography at the fault. This fault was never observed. However, the satellite imagery indicates that it dominates the topographic expression west of our fault. 

Twenty five kilometers to the west, we identified and visisted another of these thrust splays at cross-section G-G'. Based on continuous uplift, it appears to be the same fault as the one observed in the southern portion of the Haliun cross-section (F-F'). This fault was visible in outcrop. The fault strikes approximately east and dips 15 degrees. The foot wall consists of low grade metamorphosed mudstones. The hanging wall consists of similar clastic metasedimentary rocks plus greenschist. Inwards, there are multiple layers of brecciated footwall and hanging wall rock and multiple layers of fault gouge. Kinematic indicators suggest thrust sense of motion. A photograph is shown in Figure \ref{faultzones}B. The topography rises sharply directly south of the location of the fault. The basin margin elevation on this fault also decreases in the north suggesting a progressive decrease in slip.


