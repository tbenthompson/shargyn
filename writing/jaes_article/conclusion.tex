\section{Conclusion}
	The India-Eurasia collision causes deformation through a broad swath of Asia. In central Asia, the Altai Range and Gobi-Altai Range accomodate 10-15\% of the total convergence rate \citep{Calais2003}. Hence, western Mongolia has some of the best examples in the world of intracontinental deformation patterns. Here, we studied the zone where the Altai fault system and the Gobi-Altai fault system intersect. Transpressional strike-slip deformation is dominant and results in discrete fault block mountain ranges with small basins in between. The Shargyn Basin is an anomalously large basin for the region and begs explaining. 

	We describe the basin as a combination of two active structures. A left-lateral compressional stepover on the Bogd Fault forms the Khantaishir range and the Shargyn Fault. A conjugate strike-slip intersection between the Bogd Fault and the Tonhil Fault forms a wedge of uplifted material in the southwest corner of the basin. Further uplifts surround the basin due to termination thrust splays from the various strike-slip faults. In addition, the locations of these faults closely follow the local foliation, bedding, or prior faulting planes.  Thus, the structure Shargyn Basin region is a prime example of the features that occur in intraplate transpressional orogenies and the part that pre-existing structural weaknesses can play in defining their locations.
