\section{Formation of the Shargyn Basin}
	With a kinematic description of the Shargyn Basin in hand, it is interesting to ask why the Shargyn Basin exists. The Upper and Lower Khantaishir Thrusts translate a portion of the strike-slip motion north into the Shargyn Fault. Westward (left-lateral) advection of material along the Bogd Fault is transformed into uplift by the Khantaishir thrusts. However, further north, this westward motion continues. The differential motion between this westward motion and the comparatively stationary Shargyn Basin is accomodated on the Shargyn Fault and the Dariv Thrust. But, why does this stepover form at all? Moreover, why did it form in its current position? What does the answer tells us about the tectonic processes at work in the area? These are all fundamentally the same question: what causes a given fault to form in a specific location? This is a deep and difficult question to answer in the general case. However, for the Shargyn Basin, there is a partial answer.

	The first answer to this questions involves pre-existing weaknesses in the crust. Near the stepover, the Bogd Fault bends southward 20-30 degrees. For most of its length, it closely follows the Main Mongolian Lineament. The Main Mongolian Lineament is probably a suture zone and separates a Late Proterozoic and Early Paleozoic orogeny on the north side from a Late Paleozoic orogeny on the south side. The bend in the Bogd Fault may be due to a bend in the Main Mongolian Lineament. \citet{Windley2007} has mapped the Main Mongolian Lineament as bending southwards in this area as well. On a broader scale, the Gobi-Altai and Altai ranges both follow the approximate structural grain in Figure \ref{grain}.
	
	In almost all (exluding specific parts of the intersection wedge) of the locations we studied the foliation strikes parallel or close to parallel to the important fault in the region. The foliations fit well with a model where faults form along structural weaknesses. The foliations could also be explained by dragging along the fault. With enough slip, regardless of the original foliation is an area, dragging of the foliation near the fault will lead to approximately fault-parallel strike. Dragging can be clearly in the satellite imagery of the beds along the Bogd Fault in the intersection wedge. However, in other locations, foliations are still fault-parallel 5-10km away from the surface trace of the fault and dragging is not seen. So, dragging may be locally important in the intersection wedge, but, elsewhere, it is more likely that the faults formed along pre-existing structural weaknesses.

	The second answer to the question of why the Shargyn Basin formed involves the lithology of the basin basement. Based on the bouguer gravity measurements described, there is a positive density anomaly beneath the Shargyn Basin. Because identifying the exact source of a gravity anomaly is mathematically ill-posed, it is impossible to determine precisely what causes this positive density anomaly. But, it does indicate that something is different. The positive density anomaly combined with the lack of faults cutting the basin suggests a mechanically strong crustal block. The Sichuan and Tarim basins are other, larger, more well-known, examples of mechanically strong crustal blocks. 
	
	The Sichuan basin is underlain by a fast P-wave velocity zone down to 250 km \citep{Royden2008}. A very steep topographic rise surrounds the north and west margins of the topographically low basin. The Sichuan basin is up to 10km deep and ~300km E-W by 200km N-S in extent \citep{Burchfiel2008}. The Tarim basin is much larger. The Tarim basin is an Archaen and Proterozoic cratonal block, leading to its strength\citep{Lu2008}. The weak Tibetan Plateau is separated from the Tarim Basin by the Altyn Tagh Fault (see Figure \ref{asia}). The northern border of the Tarim Basin is the Tian Shan Range. The Tian Shan is especially mechanically weak, leading to the concentration of deformation. Both the Sichuan and Tarim basins show very little Mesozoic and Cenozoic deformation compared to their surroundings, indicating that their strength is not a recent development \citep{Neil1997}. The lack of active faults or topography cutting the basins is an obvious indication of their strength. The Junggar, 350km to the southwest of the Shargyn Basin, is also interpreted as a mechanically strong basin, perhaps due to a trapped piece of oceanic crust \citep{Carroll1990a}. 

	Because of the complex Paleozoic tectonic environment, it is possible for a large distinct crustal block to have accreted at the site of the Shargyn Basin. The Shargyn Basin is surrounded by uplifts, similar to the Tarim or Sichuan Basins, does not intersect region Paleozoic structures, and exhibits a gravity anomaly. As discussed, seismic velocity anomalies are too low resolution to determine anything about the Shargyn Basin. We also do not have the data to examine the Shargyn Basin for Paleozoic or Mesozoic deformation, but the largest Paleozoic structure, the Main Mongolian Lineament, in the area does not intersect the Shargyn Basin. The surface trace of the Main Mongolian Lineament follows the southern boundary of the Shargyn Basin from south of the Khantaishir Range west to the Tonhil Fault. The exact reason for this bend in the Main Mongolian Lineament is unknown, but it could indicate a regionally distinct past tectonic environment in the Shargyn Basin.

	The three earthquake locations shown in \ref{eqlocs} that are in the Shargyn Basin could shed some doubt on this hypothesis concerning the basin's underlying basement. However, none of these earthquakes were particularly large. So, because of the low seismometer density in Central Asia, the error on the locations is large. If the earthquakes are, in fact, on active faults in the Shargyn Basin, because of the lack of topographic expression, the faults cannot have accomodated a particularly large amount of uplift.

	Notably, the kinematics for the Shargyn Basin explain a large basin without any extension. Despite being deeper than most basins in the area (963m above sea level at the deepest), it would unexpected for the Shargyn Basin to have formed by any extensional process. Previous researchers have stated that the region has no normal faulting or transtensional faulting \citep{Cunningham2005a}. The low elevations do not require extension. The "Valley of the Lakes" to the north is at similar elevations. The small transtensional basins near the intersection wedge were the only extensional features we found. However, these transtensional basins are multiple orders of magnitude smaller than the Shargyn Basin. Further to the northeast, there are small normal faults linked by a major strike-slip fault \citep{Walker2006}. These features are also much smaller than the Shargyn Basin. In addition to these arguments, the low elevations could be due to the positive density anomaly. Denser crust would naturally isostatically adjust to be topographically low. 


