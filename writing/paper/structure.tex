\section{Structural Evidence}
\subsection{Remote Sensing}
Previous studies have examined Mongolian active tectonics from a remote sensing perspective \citep{Cunningham2005a}\citep{Cunningham2010}\citep{Tapponnier1979}\citep{Walker2007}. I performed a more detailed remote sensing survey of the Shargyn Basin region using Landsat 7 TM+ satellite imagery. Surface features indicating active motion were identified. Because the region is arid, many important features are clearly visible. For example, the trace of Tonhil Fault on the western margin of the Shargyn Basin is visible as a very sharp linear topographic feature. Using such features from Landsat imagery was useful in identifying important features for further investigation in the field. Using contiguous features on the satellite imagery, field mapping results can be extended to nearby, unobserved areas.

In the remote sensing studies, I focused on identifying prominent faults using steep topography, sharp color changes (lithology changes), fault scarps, and offset features. I also mapped clearly distinguishable lithologic units using color, weathering, and foliation changes visible in the Landsat imagery. Where possible, I noted the strike of the foliation. Foliation is useful to map out the Paleozoic structural grain in the region. I also noted the location and distribution of alluvial fans. Locations where alluvial fans are forming above active thrusts are important because that might indicate a temporal or slip difference between multiple adjacent faults.

\subsection{Field Mapping}
During the summer of 2012, Claire Bucholz, myself, and Esoo Erdene, a student at Mongolian University of Science and Technology performed two months of fieldwork in the Shargyn Basin region. We spent part of the time focusing on the active faults. We performed 10-20km transects through areas with important faults, as previously identified by satellite image (mostly similar to the faults identified by \citep{Walker2007}). We mapped at a broad scale, but made careful note of features indicating recent brittle deformation. Topography, fracturing, fault exposures and shifts in foliations or lithology were all important in locating faults.  Mapping was performed on paper with the assistance of satellite imagery and a GPS device. Later, the mapping was transferred to digital form.

The results of this mapping are summarized in a fault map in Figure \ref{faultmap}. Published mapping in the Sutai Range by \citet{Cunningham2003} and in the Dariv Range by \citet{Dijkstra2006} is incorporated. Further unpublished mapping of the Dariv Range by Claire Bucholz is incorporated. We performed new mapping surrounding the Shargyn Basin. Eleven cross-sections were created. Six of these are shown in the text. The remaining cross-sections are in Appendix A. The following sections describe in detail the important structural results, while referencing these cross-sections.
 
\begin{figure}[h!]
  \centering
  \includegraphics[width = 7in, height = 5.0in]{fault_map.png}
  \caption{A map of the Shargyn Basin showing the studied faults and important locations. Slip is indicated by thrust teeth or lateral slip arrows.}
  \label{faultmap}
\end{figure}


\subsection{Khantaishir Thrusts}

\begin{figure}[h!]
  \centering
  \includegraphics[width = 7in, height = 9.42in]{khantaishir_joint.png}
  \caption{A northeast-southwest cross-section through the Khantaishir Range and the Khantaishir Thrusts paired with a Landsat image. Important locations and lithology are noted on the satellite image.}
\end{figure}

The eastern edge of the basin is uplifted with motion concentrated on two faults which I call the Upper Khantaishir Thrust (UKT) and the Lower Khantaishir Thrust (LKT). The cross-section across these two faults is marked on the map as A-A'. Other minor faults are identifiable on the satellite imagery. We did not investigate the minor faults in the field. Both the UKT and LKT strike approximately NNW-SSE. The UKT is the northern fault. On the northeast side of the UKT, we observed a thick fractured layer of volcaniclastics underneath a sedimentary sequence with a thick conglomerate layer followed by shales, sandstones and carbonate reef deposits. At higher elevations in the mountain range, mafic and ultramafic ophiolite-related rocks are present. No outcrop is found on the southwest side of the UKT. At the alluvium-outcrop transition, we observed a 3-5m wide deformation zone that we interpret as the UKT itself. This zone is shown in Figure \ref{faultzones}A. Nearing this thrust trace, there is a significant increase in fracturing with strikes between 280$\circ$ and 320$\circ$. Fracture dip was similar to the dip of the UKT itself. Rotated sigmoidal tension gashes were observed immediately above the deformation zone indicating a thrust sense of motion. S-C fabric within the deformation zone also indicated thrust motion.

\begin{figure}[h!]
  \centering
  \includegraphics[width = 4in, height = 6in]{faultscollage.png}
  \caption{Photographs of exposed faults. Field workers are used as scale bars. A) An outcrop of the exposed Upper Khantaishir Thrust from cross-section E-E'. B) An outcrop of the exposed thrust in cross-section G-G'}
  \label{faultzones}
\end{figure}


The trace of the Lower Khantaishir Thrust (LKT) is located 10km to the southwest of the UKT. South from the UKT, there is a transition from recent alluvium to more lithified, potentially older alluvium. Then, for 2 kilometers before reaching the LKT, there is a thick stack of sandstone and basalt flows, with interlayered sandy carbonate beds. Finally, there is a very sharp topographic transition and lithologic shift to recent alluvium. This marks the fault trace for the LKT. The sedimentary sequence northeast of the LKT indicates the presence of a minor basin in the area at a point after the widespread metamorphism affecting the area.

Motion on the LKT may have begun after the motion on the UKT. In the small uplifted region near the trace of the LKT, there are deeply incised, but highly sinuous incised drainage canyons. Such incised drainages indicate that, before uplift began, a pre-existing sinuous drainage pattern existed. The uplift was rapid enough that the drainage pattern was "locked in". Because these drainages are aligned with the current southwestwards drainage patterns from the Khantaishir Range and the UKT, it fits with a southwestwards drainage pattern for the time-period before the LKT was active. The deeply drainages between the LKT and the UKT also indicate that the LKT is a later uplift. A major drainage cuts through the uplift adjacent to the LKT. Such a drainage would be unlikely to form unless it was present before the activation of the LKT, providing further evidence that uplift began on the UKT before beginning on the LKT.

\subsection{Bogd Fault and Termination Thrust Splays}


\begin{figure}[h!]
  \centering
  \includegraphics[width = 7in, height = 6.83in]{haliun_joint.png}
  \caption{A cross-section through the Haliun Thrust and Butara Thrust paired with a Landsat image. Important locations and lithology are noted on the satellite image.}
  \label{haliun_joint}
\end{figure}

Along the southern margin of the basin, we performed two transects across two thrusts splayed off the Bogd left-lateral strike slip fault. We will refer to the northern thrust as the Haliun Thrust and the southern thrust as the Butara Thrust. Above (south) of the Haliun Thrust, there is a complex sequence of metasedimentary and metavolcanics rocks that are intruded by a large gabbro body. There might be a fault-adjacent anticline. I do not know whether such an anticline is actively forming or a remnant from previous orogenies. The foliation in the fault-adjacent units is parallel to the Haliun Thrust and dips to the south. Along the basin margin, elevations decrease to the west. The decrease in elevation indicates the uplift tapers off further from the Bogd strike-slip fault. 

To the south of the Haliun Thrust, the Butara Thrust uplifts 3000m high peaks in the center of the Bogd Range. To the north of the Butara Thrust, we observed the same metasedimentary sequence as is adjacent to the Haliun Thrust. To the south of the Butara Thrust, there are sandstones and basaltic units. Just to the west of the cross-section, alluvium is accumulating south of the Haliun Thrust, indicating that more of the uplift is occuring on the Butara Thrust. To the west of cross-section, the high topography is almost entirely south of the Butara Thrust. 

The Butara Thrust is exposed 25km west of cross-section B-B'. Data from a transect performed at this location is shown in cross-section X1-X1' in Appendix A. Based on the outcrop, the fault strikes E-W and dips 15$\circ$. The foot wall consists of low grade metamorphosed mudstones and sandstone. The hanging wall consists of greenschist. Inwards, there are multiple layers of brecciated footwall and hanging wall rock and multiple layers of fault gouge. Kinematic indicators suggest thrust sense of motion. A photograph is shown in Figure \ref{faultzones}B. The topography rises sharply directly south of the location of the fault.

Further west, there is a third thrust splay off the Bogd Fault that can be seen in Figure \ref{faultmap}. But, we did not visit this fault in the field map. The satellite imagery suggests that it is similar in nature to the Haliun Thrust and the Butara Thrust.

\subsection{Tonhil Fault and Intersection Wedge Thrusts}
On the south-western margin of the Shargyn Basin, the wedge between the Tonhil Fault and Bogd Fault exhibits some of the most extensive thrusting. 

\begin{figure}[h!]
  \centering
  \includegraphics[width = 7in, height = 8.88in]{intersection_wedge_joint.png}
  \caption{Landsat imagery showing the intersection wedge area at the southwestern corner of the Shargyn Basin. Mapped active faults are shown by solid red lines. Slip sense is shown with thrust teeth or lateral slip arrows. Note the two conjugate strike-slip faults, the stacked thrusts at the basin margin and the two small transtensional basins located at the eastern and northern corners of the wedge. Cross sections C-C' and D-D' through the transtensional basin and the intersection wedge are shown. Cross-sections X2-X2' and X3-X3' are shown in Appendix A.}
  \label{intersection_wedge_joint}
\end{figure}

At the eastern corner of the intersection wedge, there is a bend in the Gobi-Altai Fault from E-W striking to ENE-WSW striking. At the location of this bend, there is a small (2km wide) transtensional basin, shown in cross-section C-C'. This has been previously identified on satellite imagery by \citet{Cunningham2010}. We examined the area in the field. Though we were not able to directly observe any deformation structures, the southern margin of this basin had a very steep topographic rise. North of the basin, we found a large granite intrusion. South of the basin, there are south dipping high grade metamorphic rocks. Because of the topographic depression and local deposition, the small basin is probably bounded by a normal fault on one side and the left-lateral Bogd Fault on the other. 

At cross-section D-D' (shown in Figure \ref{intersection_wedge_joint}, we performed a transect through the series of thrusts separating the basin from the intersection wedge. At the site of the transect, there are three thrusts. The lowest thrust marks the basin margin. To the southwest of this thrust, there is a large hornblende-rich diorite intrusion with minor intrusions of granite included. Close to the middle thrust, there is sedimentary sequence that includes an actively mined coal bed. The middle thrust causes a noticeable topographic jump. On the southwest side of the middle thrust, there are intrusions of gabbro and diorite. The southwesternmost thrust separates these intrusive rocks from a schist and carbonate unit. To the southwest, the topography levels out onto a broad plateau over the whole intersection wedge area. Based on the topography, each of these faults accomodates some portion of the intersection wedge uplift. However, digital elevation profiles separating the basin and the intersection wedge point to along-strike variation in how much uplift is accomodated on each of the faults. 

The western boundary of the intersection wedge is controlled by the Tonhil Fault. The Tonhil Fault is a major NNW-SSE striking right-lateral strike slip fault. At the cross-section X3-X3' (shown in Appendix A), we examine the fault where it interacts with the northwestern corner of the intersection wedge. At the latitude of the cross-section, the Tonhil fault bends so that it strikes nearly N-S. This bend appears to have created another transtensional basin. We did not examine the transtensional basin in the field. It is intriguing that it lies symmetrically across the intersection wedge from the transtensional basin on the Bogd Fault examined in cross-section C-C'. 

Further east, along cross-section X3-X3', we observed a boundary between greenschist grade metasedimentary rocks on the east side and a gneissic unit on the west side. This is interpreted as the furthest northern extent of the southwestern-most of the intersection wedge thrusts. For most of its length, satellite imagery of this fault shows some topographic expression and where observed in the field, it always separates the gneiss and schist from the metasedimentary rocks, indicating that it is an important structural boundary. 

At the far southern corner of the thrust wedge, we examined the terminus of the Bogd Fault, shown in cross-section X2-X2' in Appendix A. After the various thrust splays further east and north, it is unlikely that there is significant relative motion so far south. However, what remains is directed into a thrust fault that strikes approximately 45 degrees off the main Bogd Fault. There is a noticeable topographic rise bending away form the main Bogd Fault. The small depositional region and the topographic rise would be unlikely without recent motion. In addition, there was a higher local degree of damage to the bedrock. 

\subsection{Shargyn Fault and Termination Thrust Splays}

The northern margin of the basin is separated from the Dariv Range by the Shargyn Fault. Four of the transects focus on this fault. The fault strikes WSW-ENE. In the cross-sections, I have assumed that the Shargyn Fault is approximately vertical because of its lateral slip and the lack of other constraints. 

\begin{figure}[h!]
  \centering
  \includegraphics[width = 7in, height = 7.5in]{lil_dariv_joint.png}
  \caption{A cross-section through the Little Dariv Mountain area paired with a Landsat image of the surrounding area. Important locations and lithology are noted on the satellite image.}
  \label{lil_dariv_joint}
\end{figure}
\clearpage
The western end of the Shargyn Fault, nicknamed Little Dariv Mountain, exhibits thrusts in a small positive flower structure, seen in cross-section E-E' in Figure \ref{lil_dariv_joint}. The positive flower structure has been identified by previous researchers \citep{Howard2006}. Their interpretation is similar to the one discussed here. However, there are differences in the number and location of thrusts. To the north of the Shargyn Basin is the Dariv Basin, a much smaller, higher elevation basin. Little Dariv Mountain is located on the boundary of these two basins. The majority of the mountain is composed of greenschist-facies sedimentary and volcanic rocks. On the northern side of the range, there is a 100 meter thick band of folded carbonates and marls. Separating the range is a thrust fault. The fault follows a valley that is clearly visible on satellite imagery. To the south of the fault is a steep 200 meter topographic rise, indicating that the fault dips southwards. No other faults were identifiable in the core of Little Dariv Mountain. 

On the cross-section, two other faults are indicated on the periphery of the mountain. The southern fault can be seen in the satellite imagery as an uplifted trace of bedrock amongst a sea of basin sediments. It may be related to a large anticline visible on the satellite imagery west of Little Dariv Mountain. The northern boundary fault is the inferred continuation of the main Shargyn Fault trace.  This continuation of the Shargyn Fault terminates as thrust motion on thrusts in the Dariv Basin. Because the Dariv Basin thrusts are the easternmost extent of the Sutai Range restraining bend \citep{Cunningham2003}, the thrusting involved in the Sutai Range restraining bend can be linked with the strike-slip motion on the Shargyn Fault. 

In the central south of the Dariv Range, we examined a simple section of the Shargyn Fault, marked X4-X4' in Figure \ref{intro} and shown in Appendix A. To the north, the topography rises steeply towards a relatively flat plateau. The bedrock is composed of serpentinite, biotite-gabbro and gabbro-diorite, separated by intrusive contacts. Despite the steep topographic rise at the fault, it is unlikely that there is any dip-slip on the Shargyn Fault here. Thrusting on the Shargyn Fault at this location would conflict with the lack of topographic expression along the fault east of the Dariv Range. It is more likely that the Dariv Thrust to the northeast is uplifting the range and the Shargyn Fault is simply accomodating the juxtaposition of the uplifted Dariv Range and the Shargyn Basin. 

\begin{figure}[h!]
  \centering
  \includegraphics[width = 7in, height = 6.85in]{neshargyn_joint.png}
  \caption{A cross-section through the Shargyn Fault in the northeast corner of the Shargyn Basin paired with a Landsat image of the surrounding area. Important locations and lithology are noted on the satellite image.}
  \label{neshargyn_joint}
\end{figure}

In the northeastern part of the basin, the Shargyn Fault and a small adjacent thrust are shown in cross-section F-F' (Figure \ref{neshargyn_joint}). The small thrust uplifts the area south of the Shargyn Fault. The thrust is visible as a small, potentially recent \citep{Bayasgalan2005a}, scarp in the alluvium. The Shargyn Fault marks a major lithologic change in this area. To the south of the Shargyn Fault, we find quartz-rich greenschist, metabasalt and carbonate units with foliations dipping to the north. On the north side of the fault, there are amphibolite facies gneiss and schist units with south-dipping foliations. There are also some large granitoid intrusions. Outcrops of garnet schist confirm the higher metamorphic grade. To the west, offset stream beds shown in Figure \ref{shargynoffsetstreams} provide one of the clearest indications of the active slip sense.

\begin{figure}[h!]
  \centering
  \includegraphics[width = 4in, height = 2.0in]{shargyn_fault_streams.png}
  \caption{Offset stream beds along the ENE-WSW striking Shargyn Fault observed in Landsat imagery west of cross-section C-C'.}
  \label{shargynoffsetstreams}
\end{figure}

To the east, the strike of the Shargyn fault bends slightly southwards, at the cross-section marked X5-X5', shown in Appendix A. Here, the fault is still visible in satellite imagery as a eroded valley. To the south of the fault, we see steep rolling hills with highly deformed and metamorphosed amphibolite and gneiss. To the north of the fault, the topography is lower and transitions into alluvium. The northern unit is an slightly deformed alkali granite. The difference in metamorphic grade between the granite and the metamorphic unit indicates the granite was intruded after metamorphism. However, the contact between the granite and the metamorphic is perfectly straight, supporting the contact's interpretation as a fault. The topographic depression along the fault provides further evidence. To the east, the fault terminates somewhere near the city of Altai. 

West of the transect, a thrust splays off the Shargyn Fault striking southwards. The thrust intersects the transect in the south between a red very poorly consolidated sandy unit and a steeply dipping shale. The northern side of the thrust is uplifted, while the basin is on the south side. Elevation profiles suggest there the fault accomodates increasing amounts of uplift further south. At its southern end, this thrust connects with the Upper Khantaishir Thrust. 

\subsection{Earthquake Mechanisms and Locations}
	Earthquake focal mechanisms provide useful information about active fault slip in the region. The available calculated focal mechanisms for western Mongolia are shown in Figure \ref{broad}. The World Stress Map \citep{WorldStressMap2008} uses focal mechanisms as the exclusive constraint on the primary stress axis in western Mongolia due to the lack of other data. \citet{Bayasgalan2005a} provides a good review of the earthquake focal mechanisms in western Mongolia. Only a couple mechanisms are located in the Shargyn Basin. On the Shargyn Fault, near cross-section C-C', a focal mechanism was calculated showing exclusively left-lateral slip. In the northern corner of the intersection wedge, a focal mechanism was calculated either showing low-angle south directed thrusting or high-angle north directed thrusting. Both possibilities are reasonable given the location of the earthquake.  

\begin{figure}[h!]
  \centering
  \includegraphics[width = 5in, height = 2.88in]{eq_locs.png}
  \caption{Shaded relief map of the Shargyn Basin showing blue plus signs where earthquakes have occured between 1960 and 2013. Black lines are fault locations, like in Figure \ref{faultmap}. Note that many of the earthquakes occur on, or near, mapped faults. However, three of the earthquakes occured within the basin.}
  \label{eqlocs}
\end{figure}	

	Earthquake locations can also constrain the location of active deformation. For smaller earthquakes, focal mechanisms cannot be calculated because there is a lack of data. However, locations can still be determined, albeit with significantly decreased accuracy. The locations for all the earthquakes available from the IRIS Data Management System in the Shargyn Basin region are shown in Figure \ref{eqlocs}. Most of these earthquakes occur around the margins of the basin at locations near where we have mapped faults. Notably, there are three earthquakes shown in the eastern part of the Shargyn Basin. We lack field or satellite data constraining the location of any fault through the basin here, so I did not map a fault. Furthermore, the error on some of the earthquake locations is large.


