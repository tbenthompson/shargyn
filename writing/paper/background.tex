\section{Background on Mongolian Active Deformation}
	Understanding the tectonic context of the Shargyn Basin in western Mongolia and Asia is critical to interpreting the active deformation. 

\subsection{Mongolian Deformation in Context -- Asia}

\begin{figure}[h!]
	\centering
	\includegraphics[width = 7in, height = 6in]{asia.png}
	\caption{"Simplified tectonic map of East Asia and the Western Pacific illustrating latest Cretaceous and Cenozoic faults in East Asia. 1, 2 and 3 are major thrust, strike-slip and normal fault, respectively; 4, 5 and 6 are convergent (filled triangles are for active and open triangles are for inactive boundary), transform and divergent plate boundary, respectively; 7 is land; 8 is sea, with dark grey being basin or ocean floor and light grey being continental shelf or morphological high on basin or ocean floor. BHb, Bohaiwan basin; BSb, Banda Sea basin; CKg, Central Kamchatka graben; ECSb, East China Sea basin; FMg, Fushun-Meikehou graben; Fr, Fenwei rift; HYr, Hetao-Yinchuan rift; Jb, Java basins; MGb, Mergui basin; MLb, Malay basin; OT, Okinawa Trough; Pb, Pattani basin; Sb, Sumatra basins; SCSb, South China Sea basin; SCSmb, South China Sea marginal basin; SHdsz, Sakhalin-Hokkaido dextral shear zone; SLb, Shantar-Liziansky basin; SXr, Shanxi rift; Yb, Yunqing basin; YSb, Yellow Sea basin; YYg, Yilan-Yitong graben; Zb, Zengmu basin; Zu, Zenghe uplift" \citep{Schellart2005}. The location of Figure \ref{broad} is outlined in Central Asia. }
	\label{asia}
\end{figure}	

	Most of the active deformation in Mongolia is probably a result of the Indo-Eurasian collision \citep{Yin2010}\citep{Walker2007}\citep{Cunningham2005a}\citep{England1997}. Indo-Eurasian collisional deformation processes can be separated into "kinematic" and "dynamic" models. The basic "kinematic" model suggests that the Indo-Eurasian collision has caused eastwards extrusion of crustal blocks towards the "free" eastern boundary (the East Asian Subduction Zone)\citep{Tapponnier1982}\citep{Peltzer1996}. This hypothesis suggests that most motion has been accomodated by large strike-slip and thrust faults. The "dynamic" models propose that the viscosity of the lithosphere is more important than upper crustal faults \citep{Houseman1993}\citep{England1997}. In these models, the important effects are continuous viscous deformation of the lithosphere and the gravitational effects of the resulting thickening. 

	Rollback on the East Asian Subduction System (EASS) provides an additional explanation for active and past motions in Mongolia \citep{Schellart2005}. The EASS has been treated as a free-boundary in many of the East Asian tectonic models\citep{Tapponnier1982}\citep{Peltzer1996}\citep{Houseman1993}. But, because of its orientation, active EASS rollback would exert similarly oriented stresses to the Indo-Eurasian collision. However, the potential for far-field stresses from subduction zone rollback is not well constrained. Back-arc extension may not be limited to the deep, ocean-floor producing, back-arc basins. Much of the EASS was experiencing rollback during the late Cretaceous until the Mid-Miocene \citep{Northrup1995}\citep{Schellart2005a}. A physical model by \citet{Schellart2005} showed that, if the entire East Asian Subduction zone were to rollback, intraplate deformation could occur as far as 3000 km from the subduction zone. Additional constraints on the active and past deformation in western Mongolia could influence these broad theories of Asian deformation.

	 The intensity of intracontinental deformation in Mongolia is emphasized by measured GPS motions and three of the largest strike-slip earthquakes ever. For a region so far from a plate margin, the rate of crustal deformation in Mongolia is very high. GPS measurements (Figure \ref{broad}) show that ongoing motion is toward the north in the Altai, and mostly eastward in the Gobi-Altai\citep{Calais2003}. The average rate of motion is approximately 5 mm/year relative to a stable Eurasia as defined by \citet{Calais2003}. Multiple earthquakes with a moment magnitude of 8 or greater have occured in the last century. In July 1905, earthquakes with mangitude 7.9 and 8.4 occured on the Bolnay Fault in northern Mongolia. The 1957, magnitude ~8.0 earthquake on the Bogd Fault is one of the best exposed strike-slip surface ruptures in the world\citep{Kurushin1998} \citep{Okal1976a} \citep{Pollitz2003}. The Bogd Earthquake has been used as an analogue for potential Southern California earthquake events \citep{Bayarsayhan1996}.  The intensity of deformation makes Mongolia an ideal place to study the mechanics of intracontinental fault systems.
	
	On the broadest scale, the Altai and Gobi-Altai fault systems may be related to a sequence of regularly-spaced, NW-SE striking right-lateral strike-slip faults \citep{Yin2010} located in a belt from west Mongolia through Iran. The area is marked in the northwestern corner of Figure \ref{asia}. Many of these faults terminate in large thrust systems like the Tian Shan. One explanation invokes bookshelf-like rotation along a long left-lateral shear zone creates the right-lateral faulting. \citep{Davy1988}\citep{Bayasgalan2005a}  This model implies that the development of the Altai and Gobi-Altai is intimately linked to that of the Tian Shan. Possibly, the Altai and Gobi-Altai are similar to an earlier phase in the evolution of the Tian Shan, given the progressive northward propagation of stresses as the India-Eurasian collision proceeds.

\begin{figure}[h!]
	\centering
	\includegraphics[width = 7.0in, height = 4.09in]{broad.png}
	\caption{Shaded relief map of Mongolia with fault overlays and seismic moment tensor solutions for recorded earthquakes. Important locations or faults are labeled. HTF = Hoh-Serh-Tsagaan-Salaa Fault, BF = Bolnay Fault, BGF = Bogd Fault. The study area and location of Figure \ref{intro} is outlined in black. The difference between right-lateral transpressional strike-slip in the Altai Range and left-lateral transpressional strike-slip in the Gobi-Altai is especially important. Modified after \citet{Calais2003}}
	\label{broad}
\end{figure}

	Recent studies have examined the neotectonics of the Altai range via satellite imagery and targeted field excursions\citep{Cunningham2005a}. The region is dominated by NW-SE striking right-lateral slip. The primary features are strike-slip faults, but much of the topography is created by thrusts oriented 30 degrees off the main strike slip faults. Others are linked by a right-slip faults. In some places, the thrusts are part of a restraining bend. Major faults, GPS motions, and earthquake moment tensors are shown in Figure \ref{broad}. Pure strike-slip, oblique slip and thrust earthquake moment tensors are observed, which indicates that active motion is variable depending on the specific fault \citep{Bayasgalan2005a}. The geologic slip rate for a major transpressional fault, the Hoh-Serh-Tsagaan-Salaa Fault, is ~0.3 mm/yr of shortening and ~0.9 mm/yr of dextral slip. This constitutes ~20\% of the total strain accumulation in the Altai. \citep{Frankel2010}

	Southeast of the Altai, deformation in the Gobi-Altai range is predominantly E-W striking left-lateral slip \citep{Cunningham2010}. North and south dipping thrusts striking NW-SE follow the basement structures, whereas the strike-slip faults cut across the basement structures. The deformation is diffuse over a 250-350 km north-south extent. The range accomodate the eastwards and northeastwards deformation as measured by GPS. At the southern extent of the range, the Gobi-Tian Shan Fault is continuous from the easternmost Tian Shan through the eastern Gobi-Altai, thus supporting the connection between the Gobi-Altai and the Tian Shan. Some estimates suggest 30-40 km of Cenozoic slip on the Gobi-Tien Shan Fault System (shown as "GBTS" in Figure \ref{asia}) \citep{Cunningham2003a}. There are no slip estimates for the Bogd Fault System in the north. However, empirical relationships between fault slip and fault length suggest that the Bogd Fault System has experienced at least 10 km of slip \citep{Cowie1992}.  Earthquake moment tensors show mainly left-lateral strike-slip in the Gobi-Altai. 

	Previous researchers have suggested that the Hangay Dome in central Mongolia (Figure \ref{broad}) is a rigid block causing the formation of a pair of conjugate strike-slip faults (Altai and Gobi-Altai fault systems) \citep{Cunningham2005a}. However, recent detailed mapping work has shown the presence of large strike-slip structure cutting across the Hangay Dome \citep{Walker2007}. This indicates that the Hangay Dome is being brittly deformed in a similar fashion to the Gobi-Altai to the south, and may not be as rigid as previously thought\citep{Walker2006}\citep{Walker2007}\citep{Walker2008}.

\subsection{Shargyn Basin}

	The Shargyn basin is located at the intersection of the Altai fault system and the Gobi-Altai fault system. It is approximately 150 km in east-west extent and 100km in north-south extent. A small ephemeral lake is located in the center of the basin. The basin margin varies between 1500 meters and 2000 meters, while the center is at an elevation of 963 meters. The basin is bounded on the west by the right-lateral Tonhil Fault and on the north by the left-lateral Shargyn Fault \citep{Cunningham2003}. To the east and south, the tectonic regime is less certain. The Khantaishir Range to east has been the subject of petrological studies concerning the ophiolite located there. However, the mapping from these studies has not seriously examined the potentially active faults bounding the range to the southwest and west towards the Shargyn Basin.  The neotectonics of the Bogd Range, which bounds the Shargyn basin to the south, are also poorly understood.

\begin{figure}[h!]
	\centering
	\includegraphics[width = 7in, height = 4.2in]{intro.png}
	\caption{Landsat Imagery basemap showing important locations for this study. The large white box surrounding the Shargyn Basin is the new mapping from this study. The white area surrounding the Dariv Range indicates mapping performed by \citet{Dijkstra2006} and as a part of unpublished research by Claire Bucholz. The area surrounding the Sutai Range was mapped by \citet{Cunningham2003}. The small white boxes around the Shargyn Basin show areas that we studied in the field. The red lines are labeled cross-sections that will be discussed in the Structural Evidence section. A GPS vector near the city of Altai is shown in the northeast. Earthquake focal mechanisms in the area are also shown.}
	\label{intro}
\end{figure}
	
	
\begin{figure}[h!]
  \centering
  \includegraphics[width = 4.5in, height = 5.67in]{basins.png}
  \caption{Schematic diagrams showing faulting and deposition for common transpressional uplifts and basins.}
  \label{basintypes}
\end{figure}

Many different types of basins form in strike-slip and compressional settings. Multiple types of basins can form and interact at the intersection of fault systems \citep{Busby1995}. Some of the settings we might expect:
\begin{itemize}
	\item Foreland basins form in front of a simple thrust system. Flexural subsidence can result from the neighboring uplift. This is shown in Figure \ref{basintypes}A. 
	\item Piggyback basins form in the hinterland of a thrust fault. The adjacent uplift provides a sediment source. Shown in Figure \ref{basintypes}B
	\item Restraining bend basins form similarly to foreland basins with deposition in the foreland of thrusts, but often form on both sides of a positive flower thrust structure. Shown in Figure \ref{basintypes}C
	\item Compresional stepover basins form similarly to restraining bend basins. However, rather than a bend in the primary strike-slip system, there are two parallel strike-slip fault segments, separated by at least one thrust fault. In the compressional case, the basin is essentially a foreland basin in front of the thrust. Arguably, these fault systems are just large restraining bends. Shown in Figure \ref{basintypes}D.
	\item An intersection wedge can form at a conjugate strike-slip intersection. Because strike-slip motion on both faults must be transferred into uplift, an uplifted wedge of material forms between the two faults. Shown in Figure \ref{basintypes}E
\end{itemize}

	A goal of this paper is to describe the Shargyn Basin in these terms. A summary of the results is in the Structural Interpretation section. Nearby regions have already been investigated. Not quite bordering the Shargyn Basin, to the northwest, is the Sutai Range. The Sutai Range is the largest mountain range in the region, with an ice-capped summit above 4000m. It represents the southeastern-most uplift in the Altai. The Sutai Range is an active restraining bend in the Altai fault system\citep{Cunningham2003}\citep{Howard2006}. A stepover in the right lateral fault has resulted in progressively more thrusting. The faults bounding the range exhibit a positive flower structure such that the faults on the east dip to the west and the faults on the west dip to the east\citep{Cunningham2003}. 
	
	To the east of the Sutai Range, the Dariv Thrust uplifts the Dariv Mountains. \citet{Howard2006} describe the Dariv Basin as a piggyback basin, riding on top of the block thrust up by the Dariv Thrust. The Dariv Basin could also be described as a restraining bend basin from deposition and subsidence due to the adjacent Sutai Range. On the northern boundary of the Shargyn Basin, the Shargyn fault, a left-lateral strike slip fault, bounds the southern edge of the Dariv Mountains, separating the Shargyn Basin from the Dariv Basin.

\subsection{Bedrock Geology}
The bedrock in western Mongolia formed as part of the Central Asian Orogenic Belt. The Central Asian Orogenic Belt formed from 1000 Ma to 250 Ma as an agglomeration of island arcs, accretionary wedges, ophiolites and microcontinents in a complex multiphase development similar to what is seen today in southeast Asia \citep{Windley2007}. The broad tectonostratigraphic provinces in Mongolia are shown in Figure \ref{bedrock}.

The Shargyn Basin lies near the Main Mongolian Lineament (MML), a boundary between rocks of very different ages. North of the MML, Proterozoic and early Paleozoic units are present, whereas on the southern side of the MML, there are Silurian and Devonian gneiss, schist and felsic igneous rocks. The area is dominated by greenschist facies accretionary complex units. In the Dariv Range, arc-related igneous units are prevalent. The northern end of the Dariv Range and the bulk of the Khantaishir Range consist of ophiolite-related mafic and ultramafic units. To the south of the Main Mongolian Lineament, the units are primarily Silurian and Carboniferous arc-related metamorphic and igneous rocks. Another phase of regional magmatism may have occured during the Permian \citep{Windley2007}. Small, post-orogenic sedimentary basins are also present throughout the region.

\begin{figure}[h!]
  \centering
  \includegraphics[width = 6.4in, height = 4.5in]{bedrockfigure.png}
  \caption{Schematic tectonostratigraphic map of Mongolia. The area we studied is located in the outlined box. The Main Mongolian Lineament is the black line passing through our study region. Credit to \citet{Windley2007}}
  \label{bedrock}
\end{figure}

The structural grain of these Paleozoic orogenies could affect the location of present day faults, because the metamorphic foliation represents a major plane of weakness. The dominant structural grain of the Central Asian Orogeny is shown in Figure \ref{grain}. There is a close correspondence between the major faults and this structural fabric. 

\begin{figure}[h!]
  \centering
  \includegraphics[width = 6in, height = 3.5in]{grain.png}
  \caption{Schematic map of the dominant structural grain in western Mongolia. The study area is boxed for reference. The shaded Junggar Block is mechanically rigid. The Junggar Block may be a trapped block of oceanic crust \citep{Carroll1990a}.  The Hangay Block consists of microcontinental fragments that accreted during the early stages of the Central Asian Orogeny. It is not shaded because it is probably not mechanically rigid.  Modified after \citet{Cunningham2005a}}
  \label{grain}
\end{figure}
\clearpage

