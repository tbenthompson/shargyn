\documentclass[10pt,a4paper]{article}
\usepackage[latin1]{inputenc}
\usepackage{amsmath}
\usepackage{amsfonts}
\usepackage{amssymb}
\usepackage{graphicx}
\usepackage{natbib}
\usepackage[margin=0.75in]{geometry}
\begin{document}
\section*{Problem}
	The Altai and Gobi-Altai are two large, actively forming, mountain ranges in central Asia. The Altai are characterized by right-lateral transpressional slip, while the Gobi-Altai are characterized by left-lateral transpressional slip \citep{Cunningham2005a}. The intersection of these two fault systems provides a natural laboratory for studying complex fault behaviors. Furthermore, given the great distance to a plate boundary, the region provides a useful perspective on the far-field stresses of the Indian collision and the East Asian Subduction System. 

\begin{figure}[h!]
  \centering
  \includegraphics[width = 4.5in, height = 4in]{regional.png}
  \caption{Topographic map of Mongolia with fault overlays and seismic moment tensor solutions for recent earthquakes. The Altai, in the west, and the Gobi-Altai in the center, are labeled. Credit to \citep{Calais2003}}
\end{figure}

	At the intersection of the two fault systems, there is a Connecticut-sized basin called the Shargyn basin. The basin is surrounded by various mountain ranges. To date, no coherent explanation for the basin has been proposed. Because it is in a region dominated by compression and shows no indication of extension \citep{Cunningham2005a}, and is surrounded on all sides by mountain ranges, the formation mechanism of the basin is difficult to understand. For my senior thesis, I intend to use data collected on the active and recent deformation surrounding the Shargyn basin to gain some insight into how the basin formed. It is also possible that the work on the Shargyn basin area would inform the debate about far-field stresses in Eurasia.
	
\begin{figure}[h!]
	\centering
	\includegraphics[width = 6in, height = 4in]{local.png}
	\caption{Satellite image of the study area, showing notable features. Question marks indicate regions for which we had no reliable data prior to our fieldwork.}
\end{figure}
	
\section*{Mongolian Deformation}
	For a region so far from a plate margin, the rate of crustal deformation in Mongolia is very high. GPS measurements show that ongoing motion is to the north in the Altai, and rapidly rotates eastward in the Gobi-Altai\citep{Calais2003}. Most rates are near 5mm/yr. This motion has resulted in multiple earthquakes with a moment magnitude of 8 or greater in the last century. In July 1905, earthquakes on the Bolnay fault in northern Mongolia had ruptures with magnitudes of 7.9 and 8.4. The December 4th, 1957 earthquake on the Gobi-Altai fault is one of the best exposed surface ruptures in the world of a major strike-slip earthquake \citep{Kurushin1998} \citep{Okal1976a} \citep{Pollitz2003}. It has even been used as an analogue for potential Southern California earthquake events \citep{Bayarsayhan1996}. But, why do we get crustal deformation in Mongolia when it is so far from a plate boundary?

	The collision between India and Eurasia is generally cited as the source of the driving stresses for the faults forming the Altai and Gobi-Altai. A classic paper by \citep{Tapponnier1982} demonstrated that many of the macro-scale features of Central and Eastern Asia can be replicated by a rigid impactor (India) hitting a plasticine block (Eurasia) with a free "eastern" surface (the East Asian Subduction Zone). Most importantly, this model demonstrated that the collision of India would cause blocks of Eurasia to "extrude" to east, into the East Asian Subduction Zone.  Many of the large-scale features predicted by this model have been confirmed \citep{Schellart2005}\cite{Yin2010}. The far-field effects of such models are a major component of the recent stresses in Mongolia. 

\begin{figure}[h!]
	\centering
	\includegraphics[width = 5in, height = 5in]{asia.png}
	\caption{Tectonic map of Asia. Large arrows show direction of crustal motion. Credit to \citep{Tapponnier1982}}
\end{figure}

	However, theoretical models of crustal deformation in Eurasia suggest that the current crustal motions in Mongolia are too large to be entirely explained by the India-Eurasia collision \citep{Calais2002a}. Also, there are indications that some of the current crustal motion in Mongolia began before the India-Eurasia collision \citep{Schellart2005}. Both these problems require a different source for the driving stresses behind Mongolian crustal motion. 
	
	Another potential source of stress is rollback on the East Asian Subduction System \citep{Schellart2005}. Subduction zone rollback occurs when the boundary between the subducted plate and the overlying plate moves oceanwards -- towards the subducting plate. Because, the subduction zone "pulls" on the overlying plate, extension occurs in the back-arc region. This is often associated with the formation of small ocean basins in back-arc of a major subduction zone. The potential for far-field stresses from subduction zone roll-back are not well constrained. Back-arc extension may not be limited to the deep, ocean-floor producing, back-arc basins. A physical model by \citet{Schellart2005} showed that, if the entire East Asian Subduction zone were to rollback, then intraplate deformation could occur as far as 3000km from the subduction zone. 

\section*{Shargyn Basin Deformation}
	

	The Shargyn basin is located at the intersection of the Altai fault system and the Gobi-Altai fault system. It is a Connecticut-sized basin, bounded on the west by the right-lateral Tonhil Fault and on the north by the left-lateral Shargyn Fault \citep{Cunningham2003}. To the east and south, the tectonic regime is less certain. The  Khantaishir Range to east has been the subject of igneous petrology studies concerning the ophiolite located there. However, the mapping from these studies has not seriously examined the faults bounding the range to the southwest and west towards the Shargyn Basin.  To the south, a wide mountain range (name unknown) bounds the basin. 

	Not quite bordering the basin, to the northwest, is the Sutai Range. It is the largest mountain range in the region, with a ice-capped summit above 4000m. A significant amount of structural work has been done on the Sutai Range and the nearby Dariv Basin. The Sutai Range is an active restraining bend in the Altai fault system. A stepover in the right lateral fault has resulted in progressively more thrusting. The faults bounding the range exhibit a flower structure such that the faults on the east dip to the west and the faults on the west dip to the east\citep{Cunningham2003}. To the east, the Dariv Thrust uplifts the Dariv Mountains. \citep{Howard2006} suggest that the Dariv Basin may be a "piggyback" basin, riding on top of the block thrust up by the Dariv Thrust. South of this, the Shargyn fault, a left-lateral strike slip fault, bounds the southern edge of the Dariv Mountains. 
	
	Because of its tectonic setting and the predominance of transpressional slip, the Shargyn Basin can be classified as a intracontinental strike-slip basin \citep{Busby1995}. Transpressional basins normally result from flexural subsidence. Sufficient thrusting results in enough added tectonic loading to cause subsidence in adjacent areas, creating a basin. Such thrusting might be expected in a tight restraining bend. The deposition zone is analogous to a very small foreland basin. Often these are paired on opposite sides of a restraining bend. Sometimes, transpressional basins can result from differential extensional stresses due to moving past the curved fault. This can result in small down dropped wedges at restraining bends. The adjacent uplift due to the restraining bend provides a sediment source. Other types of basins can form where there is significant thrusting. As described above, piggyback basins sometimes form on top of young or minor thrust sheets. Although current motion is transpressional, it is possible past motion was transtensional. It is also possible that a releasing bend in a primarily transpressional setting would result in a transtensional pull apart basin. 
	
\begin{figure}[h!]
  \centering
  \includegraphics[width = 3in, height = 4in]{strikeslipbasins.png}
  \caption{The two major types of strike slip basins. The gray dot-filled area represents the basin. Credit to \citep{Busby1995}}
\end{figure}

	Mongolia lies in the distant foreland of the India-Eurasia collision. Because of this setting, it might make sense to compare the Shargyn Basin to intermontane basins, like those found in the Rocky Mountains \citep{Busby1995}. However, the scales and motions are totally different. The Rocky Mountains are massive basement cored uplifts with little strike slip motion, whereas the Altai and Gobi-Altai have significantly less thrusting and show majority strike-slip motion.
	
	Because the Shargyn basin lies near the intersection of multiple restraining bends in two fault systems with opposite motions, is adjacent to a few large mountain ranges, and has had a complex and varied history, it is probable that some complex combination of strike-slip related factors are involved in its formation. 
	
\section*{Completed Work}
	During the summer of 2012, Claire Bucholz, an MIT EAPS graduate student, and I performed two months of fieldwork in western Mongolia in the bounding mountain ranges surrounding the Shargyn Basin. We performed 5-15 mile cross-sections through areas with important faults, as previously identified by satellite image. We mapped at a broad scale, but made careful note of features indicating recent brittle deformation. We took samples at locations that might provide useful kinematic indicators. We also took samples crossing some of the major faults, possibly allowing us to determine the uplift rate using thermochronology. 

	Over the last couple months, I have taken much of the mapped data from the summer, cleaned it, and digitized it. I have added mapping work from other sources. I've also worked on cross-sections through the area.
\section*{Plans} 
	 Moving forward, I will to continue work on the map. I will add more mapping data from neighboring areas I will also create geologic cross-sections across some of the more interesting areas that we studied. To start, I plan to make a cross-section in the east-west direction and a cross-section in the north-south direction. These will provide a launching point for discussions about the kinematics of the basin. A few of the samples collected contain useful kinematic indicators, which, viewed under thin section could test hypothesis about recent fault motion. Both of these tasks will hopefully illuminate some of the recent crustal motions in the area. Based on the data collected, I will develop a kinematic explanation for how the Shargyn basin formed. Given that the Shargyn basin appears to be unique and counterintuitive, I will relate my explanation to the literature on basin formation. Learning about general basin formation concepts, and, more specifically, about how specific basins have formed, will help to understand the Shargyn Basin.

	If enough information is available, I will extrapolate the geologic history back through the Cenozoic. Using the data we collected and satellite imagery, I will look for indications that some faults were active and are no longer active. I will also look for faults that may have changed their direction of slip. I've been communicating with Devin McPhillips, a postdoc at The University of Vermont, about geomorphological records of fault motion and fault slip. Extrapolating into the recent geologic past may provide some insight into when Indian collision stresses began influencing the region and whether, prior to that, East Asian rollback-related stresses were important in the region. More generally, it will help to determine the time-evolution of the stress field in Central Asia.

	If time constraints allow for it, I would like to determine the active deformation rates across some of the faults in the region. This would be possible using interferometric synthetic aperture radar (inSAR). This is a technique that determines the change in phase of a radar wave between two similar satellite orbits. By correcting for topography, the orbits of the satellites, and a number of other variables, ground motions can be determined to the millimeter scale. If the satellite orbits are years apart, this can resolve the rate of crustal motion. Such measurements of current ground motions could help determine both which faults remain active, and the partitioning of motion across different faults with similar strike.
\section*{Potential Complications}
	As always, there is a significant possibility that I won't be able to finish everything I set out to do. It is good that there are discrete pieces to the project. With only a complete map and a few cross sections, I would still be able to provide valuable input on how the Shargyn basin formed. Because there are nearly infinite options for continuing the research, I intend to keep a relatively simple timeline. I will continue working on the research itself through March. By that point, whatever data I have will be compiled into the final thesis paper. I plan to have a draft complete by late April. 

	It is probable that I won't have enough information to extrapolate very far into the geologic past. Hopefully, some of the satellite imagery analysis will be able to supplement this. However, if I'm limited to making interpretations about the active formation of the Shargyn basin, that will be still be valuable. 
	
	I am confident that I will be able to develop a coherent and consistent kinematic description of the Shargyn basin. However, if something goes totally awry, I will still be able to fall back on a large amount of mapping work that is useful regardless. On the other hand, if I develop a kinematic description but I cannot identify any previously research basins that have similar kinematics to the Shargyn basin, then I will need to have stronger evidence to support my argument.
	
\bibliographystyle{plainnat}
\bibliography{/home/tbent/projects/library.bib}

\end{document}
