\section{Interpretation}
	The Shargyn Basin is a combination of a conjugate strike slip fault intersection and a left-lateral compressional strike-slip stepover. The Bogd Fault to the south, the Tonhil Fault to the west, the Shargyn Fault to the north and the Khantaishir Thrusts to the east are the primary faults governing the structure of the Shargyn Basin. The Bogd Fault and the Shargyn Fault are linked in a compressional stepover by the Khantaishir Thrusts. However, some motion remains on the continuation of the Bogd Fault. The remainder of this motion is transferred into multiple thrust splays, the last of which combines with terminal thrust splays from the Tonhil Fault to form an uplifted region that I will call the intersection wedge. Similar thrust splays on the Shargyn Fault uplift the Dariv Range and link with thrust faults in the Dariv Basin and the Sutai Range. Figure \ref{schematic} shows a schematic diagram of this structure.

\begin{figure}[h!]
  \centering
  \includegraphics[width = 6in, height = 3.5in]{schematic.png}
  \caption{Schematic fault map of the Shargyn Basin region. The extent is approximately the same as the extent in Figure \ref{regional}. Description of the structure is in the text. The shaded areas are basins. Unshaded areas are mountainous. The dotted lines represent the dominant foliation strike. }
  \label{schematic}
\end{figure}

	Thus, the eastern part of the Shargyn Basin is a stepover basin forming from the Khantaishir Thrusts. The western and southern part of the basin is a foreland basin from the uplifted intersection wedge and the many termination thrust splays of the Bogd Fault. On the northern boundary, the uplift of the Dariv Range is concentrated on the Dariv Thrust and the Shargyn Fault separates the low-lying basin from the adjacent mountain range while not accomodating the uplift.
	

