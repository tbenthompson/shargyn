\documentclass[10pt,a4paper]{article}
\usepackage[latin1]{inputenc}
\usepackage{amsmath}
\usepackage{amsfonts}
\usepackage{amssymb}
\usepackage{graphicx}
\usepackage{natbib}
\usepackage[margin=0.75in]{geometry}
\usepackage{setspace}
\begin{document}
\doublespacing

I tried developing inSAR of the region. I was unable to interpret the results. I'm not sure what they mean. Here are the images. If you have suggestions, I would very much appreciate them. (ADD THE IMAGES along with a similar spatial extent google earth image)

I tried using a global seismic model to get some handle on the upper mantle and lower crustal P and S wave velocities under the basin. I was unaware about how low resolution these models are. It's quite clear that something like the Shargyn Basin is too small to show up. I tried the vanderHilst group model and multiple of the S-wave models.

I tried using crustal thickness models. These are not high enough resolution for something like the Shargyn Basin... I used Gabi Laske's crust 5.1 model.

	Most of this discussion assumes there are no significant faults deforming the Shargyn basin. The surface clearly lacks any noticeable faults. However, there could be blind structures beneath the surface. I do not have the tools to examine this possibility, so I will proceed assuming that no faults deform the Shargyn Basin. 

	The absence of active extension still leaves the possibility that the Shargyn Basin is a remnant feature from a time when extension was more prevalent. Erosion rates in Mongolia have been very low for the last 150 Ma \citep{Jolivet2007}. But, given the rate of deformation in recent times, even if the Shargyn Basin were a leftover feature from a previous tectonic regime, we would expect the active faults to be deforming it. Because the faults are not deforming (uplifting) the basin, 

\end{document}

