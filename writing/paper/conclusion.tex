\section{Conclusion}
	The India-Eurasia collision causes deformation through a broad swath of Asia. In central Asia, the Altai Range and Gobi-Altai Range accomodate 10-15\% of the total convergence rate. Hence, western Mongolia has some of the best examples in the world of intracontinental deformation patterns. Here, we studied the zone where the Altai fault system and the Gobi-Altai fault system intersect. Transpressional strike-slip deformation is dominant and results in discrete fault block mountain ranges with small basins in between. The Shargyn Basin is an anomalously large basin for the region and begs explaining. 

	We describe the basin as a combination of two active structures. A left-lateral compressional stepover on the Bogd Fault forms the Khantaishir range and the Shargyn Fault. A conjugate strike-slip intersection between the Bogd Fault and the Tonhil Fault forms a wedge of uplifted material in the southwest corner of the basin. Further uplifts surround the basin due to termination thrust splays from the various strike-slip faults. 

	A density anomaly was identified underneath the Shargyn Basin using bouguer gravity anomaly data. Some difference in crustal basement may explain the density anomaly. Such crustal basement may be mechanically strong and difficult to fault. So, this unique block of crust may be controlling the deformation in the area. The lack of cross-cutting inactive faults indicates that most of the active deformation is occuring on ancient multiply-reactivated faults. Combined with the fact that the Paleozoic and Mesozoic structures in the area steer around the basin, we have further support for a unique crustal block.


